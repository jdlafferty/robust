% !TEX root = ./robust.tex

\section{Experiments}
\label{sec:experiments}

In this section we discuss experimental results that illustrate and confirm the theoretical behavior of
median regression with corrupted data. In particular, we examine the scaling behavior of the error
with respect the level of corruption as the sample size, dimension, and correlation between predictor variables
are varied.

\subsection{Reduction of the robust lasso to median regression}

We first observe how the robust lasso is equivalent to median regression for transformed data. The robust lasso is the  estimator
\begin{align*}
  \hat\beta = \argmin_\beta \left\{ \frac{1}{n}\|y - X\beta\|_1 + \lambda \|\beta\|_1\right\}.
\end{align*}
The estimator can be rewritten as
\begin{align*}
  \hat\beta &= \argmin_\beta \left\{ \frac{1}{n}\|y - X\beta\|_1 + \lambda \|\beta\|_1 \right\} \\
  &= \argmin_\beta \left\{ \|y - X\beta\|_1 + \lambda n \|\beta\|_1 \right\} \\
  &= \argmin_\beta \|\tilde y - \tilde X_\lambda \beta\|_1
\end{align*}
where $\tilde y^T = (y^T, 0_p^T)$  and $\tilde X_\lambda^T = (X^T, \lambda n I_p)$.
We take $\lambda n = c\sqrt{n \log (2p)}$ for some constant $c$.

Using this reduction, we can carry out the estimation by
median regression of $\tilde y$ on $\tilde X_\lambda$. In all of our experiments below, the \texttt{quantreg} package in $R$ is used to carry out quantile regression for quantile level $\tau = \frac{1}{2}$. We do not tune the constant $c$, simply taking $c=0.5$.
